%=====================================================
\begin{frame}{6. Уровень сложности личных связей }

\tiny
Личные связи - это отношения между сотрудниками, которые складываются помимо их совместной деятельности. Это отношения уважения, взаимной симпатии, дружеские отношения или наоборот.  В основе личных отношений лежат чувства и эмоции, которые люди испытывают по отношению друг к другу.
\smallskip

Личные связи обеспечивают сотруднику чувство защищенности и возможность получить поддержку в трудной ситуации. Также они дают возможность обсудить конфиденциальную или негативную информацию без риска ее утечки. И, наконец, именно личные связи усиливают чувство солидарности и причастности к коллективу.
\smallskip

Выделяются личные взаимные и односторонние связи. Взаимной является такая связь, при  которой оба сотрудника проявляют интерес и доверие друг к другу, выделяя партнера из общей массы коллектива в тех или иных вопросах. Одностороння связь та, при  которой один из сотрудников проявляет интерес к другому, а тот в свою очередь по каким-то причинам встречного интереса не обнаруживает. 
\smallskip

Важным для данного блока является понятие «личное лидерство», под которым мы понимаем то, в какой мере член коллектива значим (следовательно, и выбираем) для других членов коллектива в контекстах, не связанных напрямую с работой.
\bigskip

В данном разделе анализируются ответы на вопросы С5, С9.
\smallskip

\begin{itemize}

\item [С5] Если у вас возникнет сложная/тяжелая жизненная ситуация и вам нужна будет помощь, к кому из коллег вы обратитесь?

\item [С9] Кого из коллег Вы пригласите на свой день рождения к себе домой?

\end{itemize}
\end{frame}


